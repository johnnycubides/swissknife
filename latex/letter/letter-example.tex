%% Inicio del archivo `template.tex'.
%% Copyright 2006-2013 Xavier Danaux (xdanaux@gmail.com).
%
% Este trabajo puede ser distribuido o modificado bajo las 
% condiciones de la LaTeX Project Public License V1.3c, 
% disponible en http://www.latex-project.org/lppl/.
%
% Traducción al Español por Fausto M. Lagos (piratax007@protonmail.ch), 2016.


\documentclass[11pt,a4paper,sans]{moderncv}        % posibles opciones de tamaño de fuente ('10pt', '11pt' and '12pt'), papel ('a4paper', 'letterpaper', 'a5paper', 'legalpaper', 'executivepaper' and 'landscape') y familia de fuentes ('sans' and 'roman')

% temas moderncv
\moderncvstyle{casual}                             % Los estilos disponibles son 'casual' (default), 'classic', 'oldstyle' and 'banking'
\moderncvcolor{blue}                               % las opciones de color son 'blue' (default), 'orange', 'green', 'red', 'purple', 'grey' and 'black'
%\renewcommand{\familydefault}{\sfdefault}         % descomentar al inicio de la línea para definir la fuente por defecto; use '\sfdefault' para sans serif por defecto, '\rmdefault' para roman, o cualquier otro nombre de fuente instalada en sus sistema
%\nopagenumbers{}                                  % descomente para eliminar el numerado automático de las páginas en cartas de más de una página

% Codificación de carácteres
\usepackage[utf8]{inputenc}                        % Si no esta usando xelatex o lualatex, remplace por la codificación que este usando
%\usepackage{CJKutf8}                              % descomente si necesita usar CJK para escribir su carta en Chino, Japones or Koreano
\usepackage[spanish, english]{babel}			   % comentar si su carta esta escrita en un idioma diferente del Español

% Configuración de márgenes
\usepackage[scale=0.75]{geometry}
%\setlength{\hintscolumnwidth}{3cm}                % descomente si quiere modificar el ancho de columna para la fecha
%\setlength{\makecvtitlenamewidth}{10cm}           % para el estilo 'classic', si quiere forzar el ancho del nombre. la longitud es normalmente calculada para evitar sobrelapamientos con su información personal; descomente esta línea bajo su propio riesgo

% Información personal
\name{Johnny}{Cubides}
\title{Requisito para actividades de Robótica}                               % opcional, remover o comentar si no quiere que aparezca su título personal
%\address{dirección}{código postal}{país}% opcional, remover o comentar si no quiere que aparezca sus datos de ubicación; el código postal y país son argumentos que puede omitirse o pasarse vacíos
\phone[mobile]{000 111 2222}		                   % opcional, remover o comentar si no quiere incluir su número de móvil
%\phone[fixed]{3115483925}       		           % opcional, remover o comentar si no quiere incluir su número de teléfono fijo
%\phone[fax]{+3~(456)~789~012}                      % opcional, remover o comentar si no quiere incluir su número de fax
\email{user@email.dom}                               % opcional, remover o comentar si no quiere incluir su dirección de email
\homepage{www.catalejoplus.com}                         % opcional, remover o comentar si no quiere incluir su dirección web
%\extrainfo{información adicional}                  % opcional, remover o comentar si no quiere información adicional
\rhead{\includegraphics[scale=0.2]{img/catalejo.png}}
% \photo[64pt][0.4pt]{imag/catalejo.png}                        % opcional, remover o comentar si no quiere incluir su fotografía o logosímbolo; '64pt' es la algura de la imágen, 0.4pt es el grosor del cuadro al rededor de la imágen (indique 0pt para no utilizar recuadro), 'imágen' es la ubicación y nombre del archivo de imágen a incluir
%\photo{\includegraphics[width=0.3\textwidth]{imag/catalejo.png}}
\quote{Some quote}                                 % opcional, remover o comentar si no quiere una frase o cita

% para mostrar etiquetas numéricas en la bibliografía (por defecto no se muestran etiquecas); descomente las siguientes líneas solo si usa referencias bibliográficas en su carta
%\makeatletter
%\renewcommand*{\bibliographyitemlabel}{\@biblabel{\arabic{enumiv}}}
%\makeatother
%\renewcommand*{\bibliographyitemlabel}{[\arabic{enumiv}]} % Considere reemplazar la línea 44 con esta

% bibliografía con múltiples entradas
%\usepackage{multibib}
%\newcites{book,misc}{{Books},{Others}}
%----------------------------------------------------------------------------------
%            contenido
%----------------------------------------------------------------------------------
\begin{document}
%-----       carta       ---------------------------------------------------------
% Datos del destinatario
\recipient{Señor\\Milton Acero}{Coordinador de Deportes\\Colegio Gimnasio El Lago}
\date{Febrero 11 de 2019}
\opening{Cordial saludo}
\closing{Atentamente,}
\enclosure[Anexos]{Documento del Kit básco, Documento de instalación de LuaBot y LibreCAD}          % opcional, remover o comentar si no incluye anexos 
\makelettertitle{}

Para el desarrollo de las actividades referentes a las Lúdicas de robótica solicito la instalación de las
siguientes aplicaciones en la sala de computadores:

\begin{itemize}
  \item \textit{LuaBot V2}
  \item \textit{LibreCAD}
\end{itemize}

La aplicación \textit{LuaBot V2} es desarrollada por nosotros (Universidad Nacional-Catalejo) con el propósito
de programar de manera intuitiva el hardware responsable del control de mecanismos como es el caso de un robot,
tiene la ventaja de tener licencia opensource que indica que no se debe pagar por su uso. La herramienta \textit{LibreCAD}
es un producto también opensource desarrollado por la \textit{Comunidad LibreCAD} su propósito es la creación de
piezas en 2D que en nuestro caso permitirá desarrollar mecanismos sencillos de rápido prototipado con cortadoras láser.
Es importante poder instalar éstas herramientas en sala ya que el mismo formato de robótica requiere de éstos recursos.
En el anexo adjunto el correspondiente documento para la instalación de éstas aplicaciones por parte del encargado de sala de computo.

También presento adjunto el kit esencial (básico) de robótica discriminando cada elemento; éste kit permitirá al estudiante
con la ayuda de los programas instalados tener las suficientes bases para iniciar un proyecto de robótica básico. Éste kit está
pensado con un propósito general, en el momento que el estudiante decida hacer su propio proyecto es posible que requiera más módulos
de electrónica o robótica y deba adquirirlos por su cuenta, sin embargo, éste kit está diseñado para varias sesiones.

Como le comenté en nuestra entrevista, cuento con una plataforma web para alojar los proyecto realizados bajo nuestro acompañamiento
(como catalejoplus.com); para permitir que los estudiantes hagan la divulgación de sus avances y proyectos en nuestra plataforma y en la
página web del Colegio Gimnasio El Lago, se requiere que los padres de familia permitan la publicación de imágenes, vídeos o cualquier medio
que registre a los estudiantes desarrollando tales avances o proyectos. Sugiero entonces generar un \textit{formulario de autorización de uso de imagen}
por parte del colegio indicando el propósito antes mencionado para que los padres de familia de los estudiantes lo firmen estando al tanto
de las actividades a desarrollar en las Lúdicas de Robótica, así, ni el Colegio ni Catalejo (representado por mí) tendrán alguna dificultad por el
desarrollo de las mismas.

\makeletterclosing{}

\end{document}
